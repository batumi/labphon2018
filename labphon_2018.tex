\documentclass[12pt]{amsart}
\usepackage{graphicx}
\usepackage{caption}
\usepackage{subcaption}
\usepackage[margin=1in]{geometry}
\pagestyle{empty}
\pagenumbering{gobble}
\usepackage{tipa}
\usepackage[colorlinks,allcolors=blue]{hyperref}

\renewcommand{\refname}{Selected References}
%\topmargin      =0.mm      % beyond 25.mm
%\oddsidemargin  =2.mm       % beyond 25.mm
%\evensidemargin =3.mm       % beyond 25.mm
%\headheight =0.mm \headsep    =0.mm \textheight =240.mm \textwidth=160.mm
\begin{document}
\title{Variation in Georgian using large scale data collection}
\maketitle

\begin{small}
\vspace{-0.2in}
\begin{center}
\end{center}
%\vspace{0.15in}
\end{small}

Georgian is an understudied agglutinative language spoken in the Caucus
Mountains in Eastern Europe. Georgian is a low resource language which has very
little access to software and tools (such as spell checkers, dictionaries and
search engine tokenizers) which would facilitate using Georgian in its written
form. Georgian is spoken by 1.4 million speakers in the Republic of Georgia and
members of the Georgian diaspora throughout the world. While Georgian is the
national language of Georgia, most computer systems sold in Georgia are offered
in Russian or in English, and because most search engines lack support for
Georgian, users perform internet searches using Russian or English keywords
\cite{Sherouse}.


In 2014 we created open source libraries and tools to facilitate usage of the
Georgian language by Georgian speakers \cite{Dunham}. One of these tools was
Gismet, an Android application which can be used by Georgian speakers to train
their Android smartphones to recognize their speech \cite{Juhar} using
PocketSphinx \cite{Huggins}. The software was made free, public and also open
source on GitHub, a social coding site where developers can share and contribute
to source code.

Participants discover the application from the Google Play App Store. After
installing the application, they are led through a tutorial where they record two-six
utterances to train the application to their voice. The stimuli consisted of two
SMS dictations, two web searches and two legal searches. After training, users can
add additional training sentences or begin using the application anywhere in the
Android system where keyboard input is provided. The training utterances are
uploaded to a central server where they are processed using Praat and the
CMUSphinx language model toolkit \cite{Walker} to customize the acoustic model
for the speaker. The stimuli are comprised of six utterances which were elicited
durring fieldwork with three speakers in Batumi, Georgia.


Since 2014 1,000 users have used the application to train the default language
model to their voices. The resulting dataset contains only elicited training
recordings, no user defined messages are included in the dataset. In this paper
we discuss preliminary findings in the data collected and variation in the data
along two directions, correlation of prosodic variation and GPS location of the
recording, and prosodic variation across participants. In Figure A and Figure B
we see variation in prosody, Figure A exhibits a careful yet natural articulation
while Figure B exhibits an audience-less prosody which is common among
Georgian speakers while reading.


\newpage
\begin{figure}
    \centering
    \begin{subfigure}[b]{0.55\textwidth}
        \includegraphics[width=\textwidth, height=0.25\textheight]{4}
        %\caption{Digit span results}
        \label{fig:mono}
    \end{subfigure}
    ~ %add desired spacing between images, e. g. ~, \quad, \qquad, \hfill etc.
      %(or a blank line to force the subfigure onto a new line)
    \begin{subfigure}[b]{0.45\textwidth}
        \includegraphics[width=\textwidth, height=0.25\textheight]{3}
       % \caption{Corrected digit span}
        \label{fig:bil}
    \end{subfigure}
    \caption{\footnotesize{\textit{\textsc{Left:} Map of Georgia}}}\label{fig:map}
\end{figure}



\newpage
\begin{figure}
    \centering
    \begin{subfigure}[b]{0.55\textwidth}
        \includegraphics[width=\textwidth, height=0.25\textheight]{4}
        %\caption{Digit span results}
        \label{fig:mono}
    \end{subfigure}
    ~ %add desired spacing between images, e. g. ~, \quad, \qquad, \hfill etc.
      %(or a blank line to force the subfigure onto a new line)
    \begin{subfigure}[b]{0.45\textwidth}
        \includegraphics[width=\textwidth, height=0.25\textheight]{3}
       % \caption{Corrected digit span}
        \label{fig:bil}
    \end{subfigure}
    \caption{\footnotesize{\textit{\textsc{Left:} Spectrogram of "რა ტემპერატურაა დღეს?" with exagerated prosody}}}\label{fig:prosody1}
\end{figure}

\newpage
\begin{figure}
    \centering
    \begin{subfigure}[b]{0.55\textwidth}
        \includegraphics[width=\textwidth, height=0.25\textheight]{4}
        %\caption{Digit span results}
        \label{fig:mono}
    \end{subfigure}
    ~ %add desired spacing between images, e. g. ~, \quad, \qquad, \hfill etc.
      %(or a blank line to force the subfigure onto a new line)
    \begin{subfigure}[b]{0.45\textwidth}
        \includegraphics[width=\textwidth, height=0.25\textheight]{3}
       % \caption{Corrected digit span}
        \label{fig:bil}
    \end{subfigure}
    \caption{\footnotesize{\textit{\textsc{Left:} Spectrogram of "რა ტემპერატურაა დღეს?" with natural prosody}}}\label{fig:prosody2}
\end{figure}



\bibliographystyle{IEEEtran}

\begin{thebibliography}{17}

\bibitem[1]{Dunham}Dunham, J., Chiodo, G., \& Horner, J. 2014. LingSync \& the Online Linguistic Database: New Models for the Collection and Management of Data for Language Communities, Linguists and Language Learners. \emph{Proceedings of the 2014 Workshop on the Use of Computational Methods in the Study of Endangered Languages}, 1, 23--33.


\bibitem[2]{Juhar}Juh{\'a}r, J., Sta{\v{s}}, J., \& Hl{\'a}dek, D. 2012. Recent progress in development of language model for Slovak large vocabulary continuous speech recognition. \emph{InTech: New technologies-trends, innovations and research}.


\bibitem[3]{Huggins}Huggins-Daines, D., Kumar, M., Chan, A., Black, A.,  Ravishankar, M., \& Rudnicky,  A. I. 2006. PocketSphinx: A free, real-time continuous speech recognition system for hand-held devices. \emph{Proceedings of ICASSP}.


\bibitem[4]{Sherouse}Sherouse, P. 2014. Hazardous digits: Telephone keypads and Russian numbers in Tbilisi, Georgia. \emph{Language \& Communication}, 37, 1--11.


\bibitem[5]{Walker}Walker, W., Lamere, P., Kwok, P., Raj, B., Singh, R., Gouvea, E., Wolf, P. \& Woelfel, J. 2004. Sphinx-4: A flexible open source framework for speech recognition. \emph{Sun Microsystems, Inc}.


\end{thebibliography}

\end{document}
